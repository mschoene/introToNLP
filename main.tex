\documentclass[11pt]{article}

% Language setting
% Replace `english' with e.g. `spanish' to change the document language
\usepackage[english]{babel}

% Set page size and margins
% Replace `letterpaper' with`a4paper' for UK/EU standard size
\usepackage[a4paper,top=2cm,bottom=2cm,left=3cm,right=3cm,marginparwidth=1.75cm]{geometry}

% Useful packages
\usepackage{amsmath}
\usepackage{graphicx}
\usepackage[colorlinks=true, allcolors=blue]{hyperref}
\usepackage[most]{tcolorbox}

\title{Title \\

\begin{tcolorbox}[colback=green!5!white,colframe=green!75!black,title=Example]
  Predicting the Price of Food from Restaurant Menus
\end{tcolorbox}

}
\author{Mathias Müller}

\begin{document}
\maketitle

\begin{abstract}
The abstract is a summary of the entire report in just a few sentences.
\end{abstract}

\section{Introduction}

\begin{itemize}
    \item What problem did I work on?
    \item Which major area of NLP is this about?
\end{itemize}

\begin{tcolorbox}[colback=green!5!white,colframe=green!75!black,title=Example]
  I was working on predicting the price of dishes from the text of restaurant menus.
  Overall, this is a regression problem that has text as input.
\end{tcolorbox}

\section{Data set}

\begin{itemize}
    \item What data set did I choose?
    \item How was this data set constructed?
    \item Which information is included, which meta data?
    \item Why did I choose this dataset?
    \item What would have been other options?
\end{itemize}

\begin{tcolorbox}[colback=green!5!white,colframe=green!75!black,title=Example]
  I chose the \textit{Uber Eats Menus} data set.
  The dataset was constructed by scraping the menu text and prices of food advertised by restaurants on Uber Eats.
  Each menu item also comes with a restaurant ID and user reviews.
  I think this is the best dataset for this task because it is the biggest and most diverse dataset.
  I could have used the \textit{Open Recipes Data Dump} instead, which would have made the task about prediction from the recipe text instead of the menu description.
\end{tcolorbox}

\section{Preprocessing}

\begin{itemize}
    \item Did I apply any auxiliary NLP tasks to preprocess the data?
    \item If yes, why?
    \item If yes, how exactly, using which tools?
\end{itemize}

\begin{tcolorbox}[colback=green!5!white,colframe=green!75!black,title=Example]
  I applied the preprocessing that spacy applies to input text by default. This means sentence segmentation, tokenization, ...
  Then I represented each input document as the average of the embeddings for the words it contains.
\end{tcolorbox}

\section{Model training}

\begin{itemize}
    \item What model did I train?
    \item Why did I choose this kind of model?
\end{itemize}

\begin{tcolorbox}[colback=green!5!white,colframe=green!75!black,title=Example]
  I trained a linear regression model with scikit-learn.
  I think this model is well-suited for a regression task and it is simple to interpret what the model learns.
\end{tcolorbox}

\section{Evaluation}

\begin{itemize}
    \item How did I evaluate my model?
    \item Why did I choose exactly this evaluation method?
    \item What is the main outcome of the evaluation?
\end{itemize}

\begin{tcolorbox}[colback=green!5!white,colframe=green!75!black,title=Example]
  I evaluated my model with mean absolute error.
  I think that is a standard evaluation metric for regression tasks, and I believe it is the most simple to understand.
  The mean absolute error of my model is 4.5, meaning that on average, the price predicted by the model is off by 4.5 dollars.
\end{tcolorbox}

\section{Future work}

\begin{itemize}
    \item How could my work be extended in the future?
\end{itemize}

\begin{tcolorbox}[colback=green!5!white,colframe=green!75!black,title=Example]
  In the future, someone could extend my work by also predicting the average rating from user reviews, or seeing if the price and review score are related.
\end{tcolorbox}

\clearpage

\section{Additional ideas for sections (not mandatory)}

\subsection{Levels of linguistic organization}

\begin{itemize}
    \item Which areas of linguistics matter for this problem?
\end{itemize}

\subsection{Kinds of machine learning}

\begin{itemize}
    \item What kind of machine learning is involved?
    \item Did I do a data split myself?
    \item Did I have to choose hyperparameters?
\end{itemize}

\subsection{Representations of language}

\begin{itemize}
    \item Am I using a sequence-to-sequence model?
    \item Are there embeddings in my model?
\end{itemize}

\subsection{Current challenges}

\begin{itemize}
    \item What is currently challenging in this area?
    \item How can the state of diversity be summarized in this area?
\end{itemize}

\end{document}